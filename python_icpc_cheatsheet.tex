\documentclass[11pt,a4paper]{article}
\usepackage[utf8]{inputenc}
\usepackage[margin=2cm]{geometry}
\usepackage{listings}
\usepackage{xcolor}
\usepackage{amsmath}
\usepackage{enumitem}
\usepackage{titlesec}
\usepackage{hyperref}

% Compact spacing
\setlength{\parindent}{0pt}
\setlength{\parskip}{4pt}
\titlespacing*{\section}{0pt}{8pt}{4pt}
\titlespacing*{\subsection}{0pt}{6pt}{3pt}

% Code listing style
\lstset{
    language=Python,
    basicstyle=\ttfamily\small,
    keywordstyle=\color{blue}\bfseries,
    commentstyle=\color{gray}\itshape,
    stringstyle=\color{red},
    numbers=left,
    numberstyle=\tiny\color{gray},
    breaklines=true,
    showstringspaces=false,
    tabsize=4,
    frame=single,
    frameround=tttt,
    columns=flexible,
    keepspaces=true
}

% Hyperref setup
\hypersetup{
    colorlinks=true,
    linkcolor=blue,
    filecolor=magenta,      
    urlcolor=cyan,
    pdftitle={Python ICPC Cheatsheet},
    pdfauthor={ICPC Contestant},
}

\begin{document}

\begin{center}
    {\LARGE\bfseries Python ICPC Cheatsheet}\\[0.5em]
    {\large Comprehensive Reference for Competitive Programming}\\[0.5em]
    \today
\end{center}

\tableofcontents
\newpage

% Include all sections
\section{Input/Output}

\textbf{Description:} Efficient input/output is crucial in competitive programming, especially for problems with large datasets. Using \texttt{sys.stdin.readline} is significantly faster than the default \texttt{input()} function.

\begin{lstlisting}
# Fast I/O - Essential for large inputs
import sys
input = sys.stdin.readline

# Read single integer
n = int(input())

# Read multiple integers on one line
a, b = map(int, input().split())

# Read array of integers
arr = list(map(int, input().split()))

# Read strings (strip to remove trailing newline)
s = input().strip()
words = input().split()

# Multiple test cases pattern
t = int(input())
for _ in range(t):
    # process each test case

# Print without newline
print(x, end=' ')

# Formatted output with precision
print(f"{x:.6f}")  # 6 decimal places
\end{lstlisting}

\newpage

\section{Basic Data Structures}

\subsection{List Operations}
\textbf{Description:} Python lists are dynamic arrays with O(1) amortized append and O(n) insert/delete at arbitrary positions.

\begin{lstlisting}
# Initialize lists
arr = [0] * n  # n zeros
matrix = [[0] * m for _ in range(n)]  # Correct way!

# List comprehension - concise and efficient
squares = [x**2 for x in range(n)]
evens = [x for x in arr if x % 2 == 0]

# Sorting - O(n log n)
arr.sort()  # in-place, modifies arr
arr.sort(reverse=True)  # descending
arr.sort(key=lambda x: (x[0], -x[1]))  # custom
sorted_arr = sorted(arr)  # returns new list

# Binary search in sorted array
from bisect import bisect_left, bisect_right
idx = bisect_left(arr, x)   # leftmost position
idx = bisect_right(arr, x)  # rightmost position

# Common operations
arr.append(x)      # O(1) amortized
arr.pop()          # O(1) - remove last
arr.pop(0)         # O(n) - remove first (slow!)
arr.reverse()      # O(n) - in-place
arr.count(x)       # O(n) - count occurrences
arr.index(x)       # O(n) - first occurrence
\end{lstlisting}

\subsection{Deque (Double-ended Queue)}
\textbf{Description:} Deque provides O(1) append and pop from both ends, making it ideal for sliding window problems and implementing queues/stacks efficiently.

\begin{lstlisting}
from collections import deque
dq = deque()

# O(1) operations on both ends
dq.append(x)       # add to right
dq.appendleft(x)   # add to left
dq.pop()           # remove from right
dq.popleft()       # remove from left

# Sliding window maximum - O(n)
# Maintains decreasing order of elements
def sliding_max(arr, k):
    dq = deque()  # stores indices
    result = []
    
    for i in range(len(arr)):
        # Remove indices outside window
        while dq and dq[0] < i - k + 1:
            dq.popleft()
        
        # Remove smaller elements (not useful)
        while dq and arr[dq[-1]] < arr[i]:
            dq.pop()
        
        dq.append(i)
        if i >= k - 1:
            result.append(arr[dq[0]])
    
    return result
\end{lstlisting}

\subsection{Heap (Priority Queue)}
\textbf{Description:} Python's heapq implements a min-heap. For max-heap, negate values. Useful for finding k-th largest/smallest, Dijkstra's algorithm, and scheduling problems.

\begin{lstlisting}
import heapq

# Min heap (default)
heap = []
heapq.heappush(heap, x)        # O(log n)
min_val = heapq.heappop(heap)  # O(log n)
min_val = heap[0]              # O(1) peek

# Max heap - negate values
heapq.heappush(heap, -x)
max_val = -heapq.heappop(heap)

# Convert list to heap in-place - O(n)
heapq.heapify(arr)

# K largest/smallest - O(n log k)
k_largest = heapq.nlargest(k, arr)
k_smallest = heapq.nsmallest(k, arr)

# Custom comparator using tuples
# Compares first element, then second, etc.
heapq.heappush(heap, (priority, item))
\end{lstlisting}

\subsection{Dictionary \& Counter}
\textbf{Description:} Hash maps with O(1) average case insert/lookup. Counter is specialized for counting occurrences.

\begin{lstlisting}
from collections import defaultdict, Counter

# defaultdict - provides default value
graph = defaultdict(list)  # empty list default
count = defaultdict(int)   # 0 default

# Counter - count elements efficiently
cnt = Counter(arr)
cnt['x'] += 1
most_common = cnt.most_common(k)  # k most frequent

# Dictionary operations
d = {}
d.get(key, default_val)
d.setdefault(key, default_val)
for k, v in d.items():
    pass
\end{lstlisting}

\subsection{Set Operations}
\textbf{Description:} Hash sets provide O(1) membership testing and set operations.

\begin{lstlisting}
s = set()
s.add(x)         # O(1)
s.remove(x)      # O(1), KeyError if not exists
s.discard(x)     # O(1), no error if not exists

# Set operations - all O(n)
a | b   # union
a & b   # intersection
a - b   # difference
a ^ b   # symmetric difference

# Ordered set workaround
from collections import OrderedDict
oset = OrderedDict.fromkeys([])
\end{lstlisting}

\newpage

\section{String Operations}

\textbf{Description:} Strings in Python are immutable. For building strings, use list and join for O(n) complexity instead of repeated concatenation which is O(n²).

\begin{lstlisting}
# Common string methods
s.lower(), s.upper()
s.strip()   # remove whitespace both ends
s.lstrip()  # remove left whitespace
s.rstrip()  # remove right whitespace
s.split(delimiter)
delimiter.join(list)
s.replace(old, new)
s.startswith(prefix)
s.endswith(suffix)
s.isdigit(), s.isalpha(), s.isalnum()

# String building - EFFICIENT O(n)
result = []
for x in data:
    result.append(str(x))
s = ''.join(result)

# String concatenation - SLOW O(n^2)
# s = ""
# for x in data:
#     s += str(x)  # Don't do this!

# ASCII values
ord('a')  # 97
chr(97)   # 'a'

# String to character array (for mutations)
chars = list(s)
chars[0] = 'x'
s = ''.join(chars)
\end{lstlisting}

\newpage

\section{Mathematics}

\subsection{Basic Math Operations}
\begin{lstlisting}
import math

# Common functions
math.ceil(x), math.floor(x)
math.gcd(a, b)      # Greatest common divisor
math.lcm(a, b)      # Python 3.9+
math.sqrt(x)
math.log(x), math.log2(x), math.log10(x)

# Powers
x ** y
pow(x, y, mod)  # (x^y) % mod - efficient modular exp

# Infinity
float('inf'), float('-inf')

# Custom GCD using Euclidean algorithm - O(log min(a,b))
def gcd(a, b):
    while b:
        a, b = b, a % b
    return a

def lcm(a, b):
    return a * b // gcd(a, b)
\end{lstlisting}

\subsection{Combinatorics}
\textbf{Description:} Compute combinations and permutations. For modular arithmetic, compute factorial arrays and use modular inverse.

\begin{lstlisting}
from math import factorial, comb, perm

# nCr (combinations) - "n choose r"
comb(n, r)  # Built-in Python 3.8+

# nPr (permutations)
perm(n, r)  # Built-in Python 3.8+

# Manual nCr implementation
def ncr(n, r):
    if r > n: return 0
    r = min(r, n - r)  # Optimization: C(n,r) = C(n,n-r)
    num = den = 1
    for i in range(r):
        num *= (n - i)
        den *= (i + 1)
    return num // den

# Precompute factorials with modulo
MOD = 10**9 + 7
def modfact(n):
    fact = [1] * (n + 1)
    for i in range(1, n + 1):
        fact[i] = fact[i-1] * i % MOD
    return fact

# Modular combination using precomputed factorials
# First precompute inverse factorials
def compute_inv_factorials(n, mod):
    fact = modfact(n)
    inv_fact = [1] * (n + 1)
    inv_fact[n] = pow(fact[n], mod - 2, mod)
    for i in range(n - 1, -1, -1):
        inv_fact[i] = inv_fact[i + 1] * (i + 1) % mod
    return fact, inv_fact

def modcomb(n, r, fact, inv_fact, mod):
    if r > n or r < 0: return 0
    return fact[n] * inv_fact[r] % mod * inv_fact[n-r] % mod
\end{lstlisting}

\newpage

\section{Number Theory}

\textbf{Description:} Essential algorithms for problems involving primes, modular arithmetic, and divisibility.

\subsection{Modular Arithmetic}
\begin{lstlisting}
# Modular inverse using Fermat's Little Theorem
# Only works when mod is prime
# a^(-1) = a^(mod-2) (mod p)
def modinv(a, mod):
    return pow(a, mod - 2, mod)

# Extended Euclidean Algorithm
# Returns (gcd, x, y) where ax + by = gcd(a,b)
# Can find modular inverse for any coprime a,mod
def extgcd(a, b):
    if b == 0:
        return a, 1, 0
    g, x1, y1 = extgcd(b, a % b)
    x = y1
    y = x1 - (a // b) * y1
    return g, x, y
\end{lstlisting}

\subsection{Sieve of Eratosthenes}
\textbf{Description:} Find all primes up to n in O(n log log n) time. Memory: O(n).

\begin{lstlisting}
def sieve(n):
    is_prime = [True] * (n + 1)
    is_prime[0] = is_prime[1] = False
    
    for i in range(2, int(n**0.5) + 1):
        if is_prime[i]:
            # Mark multiples as composite
            for j in range(i*i, n + 1, i):
                is_prime[j] = False
    
    return is_prime

# Get list of primes
primes = [i for i in range(n+1) if is_prime[i]]
\end{lstlisting}

\subsection{Prime Factorization}
\textbf{Description:} Decompose n into prime factors in O(sqrt(n)) time.

\begin{lstlisting}
def factorize(n):
    factors = []
    d = 2
    
    # Check divisors up to sqrt(n)
    while d * d <= n:
        while n % d == 0:
            factors.append(d)
            n //= d
        d += 1
    
    # If n > 1, it's a prime factor
    if n > 1:
        factors.append(n)
    
    return factors

# Get prime factors with counts
from collections import Counter
def prime_factor_counts(n):
    return Counter(factorize(n))
\end{lstlisting}

\newpage

\section{Graph Algorithms}

\subsection{Graph Representation}
\textbf{Description:} Adjacency list is most common for sparse graphs. Use defaultdict for convenience.

\begin{lstlisting}
from collections import defaultdict, deque

# Unweighted graph
graph = defaultdict(list)
for _ in range(m):
    u, v = map(int, input().split())
    graph[u].append(v)
    graph[v].append(u)  # for undirected

# Weighted graph - store (neighbor, weight) tuples
graph[u].append((v, weight))
\end{lstlisting}

\subsection{BFS (Breadth-First Search)}
\textbf{Description:} Explores graph level by level. Finds shortest path in unweighted graphs. Time: O(V+E), Space: O(V).

\begin{lstlisting}
def bfs(graph, start):
    visited = set([start])
    queue = deque([start])
    dist = {start: 0}
    
    while queue:
        node = queue.popleft()
        
        for neighbor in graph[node]:
            if neighbor not in visited:
                visited.add(neighbor)
                queue.append(neighbor)
                dist[neighbor] = dist[node] + 1
    
    return dist

# Grid BFS - common in maze/path problems
def grid_bfs(grid, start):
    n, m = len(grid), len(grid[0])
    visited = [[False] * m for _ in range(n)]
    queue = deque([start])
    visited[start[0]][start[1]] = True
    
    # 4 directions: right, down, left, up
    dirs = [(0,1), (1,0), (0,-1), (-1,0)]
    
    while queue:
        x, y = queue.popleft()
        
        for dx, dy in dirs:
            nx, ny = x + dx, y + dy
            
            # Check bounds and validity
            if (0 <= nx < n and 0 <= ny < m 
                and not visited[nx][ny] 
                and grid[nx][ny] != '#'):
                
                visited[nx][ny] = True
                queue.append((nx, ny))
\end{lstlisting}

\subsection{DFS (Depth-First Search)}
\textbf{Description:} Explores as far as possible along each branch. Used for connectivity, cycles, topological sort. Time: O(V+E), Space: O(V).

\begin{lstlisting}
# Recursive DFS
def dfs(graph, node, visited):
    visited.add(node)
    
    for neighbor in graph[node]:
        if neighbor not in visited:
            dfs(graph, neighbor, visited)

# Iterative DFS using stack
def dfs_iterative(graph, start):
    visited = set()
    stack = [start]
    
    while stack:
        node = stack.pop()
        
        if node not in visited:
            visited.add(node)
            
            for neighbor in graph[node]:
                if neighbor not in visited:
                    stack.append(neighbor)

# Cycle detection in undirected graph
def has_cycle(graph, n):
    visited = [False] * n
    
    def dfs(node, parent):
        visited[node] = True
        
        for neighbor in graph[node]:
            if not visited[neighbor]:
                if dfs(neighbor, node):
                    return True
            # Back edge to non-parent = cycle
            elif neighbor != parent:
                return True
        
        return False
    
    # Check all components
    for i in range(n):
        if not visited[i]:
            if dfs(i, -1):
                return True
    
    return False

# Cycle detection in directed graph
def has_cycle_directed(graph, n):
    WHITE, GRAY, BLACK = 0, 1, 2
    color = [WHITE] * n
    
    def dfs(node):
        color[node] = GRAY
        
        for neighbor in graph[node]:
            if color[neighbor] == GRAY:
                return True  # Back edge = cycle
            if color[neighbor] == WHITE:
                if dfs(neighbor):
                    return True
        
        color[node] = BLACK
        return False
    
    for i in range(n):
        if color[i] == WHITE:
            if dfs(i):
                return True
    return False

# Connected components count
def count_components(graph, n):
    visited = [False] * n
    count = 0
    
    def dfs(node):
        visited[node] = True
        for neighbor in graph[node]:
            if not visited[neighbor]:
                dfs(neighbor)
    
    for i in range(n):
        if not visited[i]:
            dfs(i)
            count += 1
    
    return count

# Bipartite check (2-coloring)
def is_bipartite(graph, n):
    color = [-1] * n
    
    def bfs(start):
        from collections import deque
        queue = deque([start])
        color[start] = 0
        
        while queue:
            node = queue.popleft()
            
            for neighbor in graph[node]:
                if color[neighbor] == -1:
                    color[neighbor] = 1 - color[node]
                    queue.append(neighbor)
                elif color[neighbor] == color[node]:
                    return False
        
        return True
    
    for i in range(n):
        if color[i] == -1:
            if not bfs(i):
                return False
    
    return True
\end{lstlisting}

\subsection{Strongly Connected Components (SCC)}
\textbf{Description:} Find all SCCs in directed graph using Tarjan's algorithm. Time: O(V+E).

\begin{lstlisting}
def tarjan_scc(graph, n):
    index_counter = [0]
    stack = []
    lowlink = [0] * n
    index = [0] * n
    on_stack = [False] * n
    index_initialized = [False] * n
    sccs = []
    
    def strongconnect(v):
        index[v] = index_counter[0]
        lowlink[v] = index_counter[0]
        index_counter[0] += 1
        index_initialized[v] = True
        stack.append(v)
        on_stack[v] = True
        
        for w in graph[v]:
            if not index_initialized[w]:
                strongconnect(w)
                lowlink[v] = min(lowlink[v], lowlink[w])
            elif on_stack[w]:
                lowlink[v] = min(lowlink[v], index[w])
        
        if lowlink[v] == index[v]:
            scc = []
            while True:
                w = stack.pop()
                on_stack[w] = False
                scc.append(w)
                if w == v:
                    break
            sccs.append(scc)
    
    for v in range(n):
        if not index_initialized[v]:
            strongconnect(v)
    
    return sccs
\end{lstlisting}

\subsection{Bridges and Articulation Points}
\textbf{Description:} Find critical edges (bridges) and vertices (articulation points). Time: O(V+E).

\begin{lstlisting}
def find_bridges(graph, n):
    visited = [False] * n
    disc = [0] * n
    low = [0] * n
    parent = [-1] * n
    time = [0]
    bridges = []
    
    def dfs(u):
        visited[u] = True
        disc[u] = low[u] = time[0]
        time[0] += 1
        
        for v in graph[u]:
            if not visited[v]:
                parent[v] = u
                dfs(v)
                low[u] = min(low[u], low[v])
                
                # Bridge condition
                if low[v] > disc[u]:
                    bridges.append((u, v))
            elif v != parent[u]:
                low[u] = min(low[u], disc[v])
    
    for i in range(n):
        if not visited[i]:
            dfs(i)
    
    return bridges

def find_articulation_points(graph, n):
    visited = [False] * n
    disc = [0] * n
    low = [0] * n
    parent = [-1] * n
    time = [0]
    ap = set()
    
    def dfs(u):
        children = 0
        visited[u] = True
        disc[u] = low[u] = time[0]
        time[0] += 1
        
        for v in graph[u]:
            if not visited[v]:
                children += 1
                parent[v] = u
                dfs(v)
                low[u] = min(low[u], low[v])
                
                # Articulation point conditions
                if parent[u] == -1 and children > 1:
                    ap.add(u)
                if parent[u] != -1 and low[v] >= disc[u]:
                    ap.add(u)
            elif v != parent[u]:
                low[u] = min(low[u], disc[v])
    
    for i in range(n):
        if not visited[i]:
            dfs(i)
    
    return list(ap)
\end{lstlisting}

\subsection{Lowest Common Ancestor (LCA)}
\textbf{Description:} Find LCA of two nodes in a tree. Binary lifting preprocessing: O(n log n), Query: O(log n).

\begin{lstlisting}
class LCA:
    def __init__(self, graph, root, n):
        self.n = n
        self.LOG = 20  # log2(n) + 1
        self.parent = [[-1] * self.LOG for _ in range(n)]
        self.depth = [0] * n
        
        # DFS to set parent and depth
        visited = [False] * n
        
        def dfs(node, par, d):
            visited[node] = True
            self.parent[node][0] = par
            self.depth[node] = d
            
            for neighbor in graph[node]:
                if not visited[neighbor]:
                    dfs(neighbor, node, d + 1)
        
        dfs(root, -1, 0)
        
        # Binary lifting preprocessing
        for j in range(1, self.LOG):
            for i in range(n):
                if self.parent[i][j-1] != -1:
                    self.parent[i][j] = self.parent[
                        self.parent[i][j-1]][j-1]
    
    def lca(self, u, v):
        # Make u deeper
        if self.depth[u] < self.depth[v]:
            u, v = v, u
        
        # Bring u to same level as v
        diff = self.depth[u] - self.depth[v]
        for i in range(self.LOG):
            if (diff >> i) & 1:
                u = self.parent[u][i]
        
        if u == v:
            return u
        
        # Binary search for LCA
        for i in range(self.LOG - 1, -1, -1):
            if self.parent[u][i] != self.parent[v][i]:
                u = self.parent[u][i]
                v = self.parent[v][i]
        
        return self.parent[u][0]
    
    def dist(self, u, v):
        # Distance between two nodes
        l = self.lca(u, v)
        return self.depth[u] + self.depth[v] - 2 * self.depth[l]
\end{lstlisting}

\newpage

\section{Shortest Path Algorithms}

\subsection{Dijkstra's Algorithm}
\textbf{Description:} Finds shortest paths from a source to all vertices in weighted graphs with non-negative edges. Time: O((V+E) log V) with heap.

\begin{lstlisting}
import heapq

def dijkstra(graph, start, n):
    # Initialize distances to infinity
    dist = [float('inf')] * n
    dist[start] = 0
    
    # Min heap: (distance, node)
    heap = [(0, start)]
    
    while heap:
        d, node = heapq.heappop(heap)
        
        # Skip if already processed with better distance
        if d > dist[node]:
            continue
        
        # Relax edges
        for neighbor, weight in graph[node]:
            new_dist = dist[node] + weight
            
            if new_dist < dist[neighbor]:
                dist[neighbor] = new_dist
                heapq.heappush(heap, (new_dist, neighbor))
    
    return dist

# Path reconstruction
def dijkstra_with_path(graph, start, n):
    dist = [float('inf')] * n
    parent = [-1] * n
    dist[start] = 0
    heap = [(0, start)]
    
    while heap:
        d, node = heapq.heappop(heap)
        if d > dist[node]:
            continue
        
        for neighbor, weight in graph[node]:
            new_dist = dist[node] + weight
            if new_dist < dist[neighbor]:
                dist[neighbor] = new_dist
                parent[neighbor] = node
                heapq.heappush(heap, (new_dist, neighbor))
    
    return dist, parent

def reconstruct_path(parent, target):
    path = []
    while target != -1:
        path.append(target)
        target = parent[target]
    return path[::-1]
\end{lstlisting}

\subsection{Bellman-Ford Algorithm}
\textbf{Description:} Finds shortest paths with negative edges. Detects negative cycles. Time: O(VE).

\begin{lstlisting}
def bellman_ford(edges, n, start):
    # edges = [(u, v, weight), ...]
    dist = [float('inf')] * n
    dist[start] = 0
    
    # Relax edges n-1 times
    for _ in range(n - 1):
        for u, v, w in edges:
            if dist[u] != float('inf') and \
               dist[u] + w < dist[v]:
                dist[v] = dist[u] + w
    
    # Check for negative cycles
    for u, v, w in edges:
        if dist[u] != float('inf') and \
           dist[u] + w < dist[v]:
            return None  # Negative cycle exists
    
    return dist
\end{lstlisting}

\subsection{Floyd-Warshall Algorithm}
\textbf{Description:} All-pairs shortest paths. Works with negative edges (no negative cycles). Time: O(V³).

\begin{lstlisting}
def floyd_warshall(n, edges):
    # Initialize distance matrix
    dist = [[float('inf')] * n for _ in range(n)]
    
    for i in range(n):
        dist[i][i] = 0
    
    for u, v, w in edges:
        dist[u][v] = min(dist[u][v], w)
    
    # Dynamic programming
    for k in range(n):  # Intermediate vertex
        for i in range(n):
            for j in range(n):
                dist[i][j] = min(dist[i][j], 
                                dist[i][k] + dist[k][j])
    
    return dist

# Check for negative cycle
def has_negative_cycle(dist, n):
    for i in range(n):
        if dist[i][i] < 0:
            return True
    return False
\end{lstlisting}

\subsection{Minimum Spanning Tree}

\subsubsection{Kruskal's Algorithm}
\textbf{Description:} MST using Union-Find. Sort edges by weight. Time: O(E log E).

\begin{lstlisting}
def kruskal(n, edges):
    # edges = [(weight, u, v), ...]
    edges.sort()  # Sort by weight
    
    uf = UnionFind(n)
    mst_weight = 0
    mst_edges = []
    
    for weight, u, v in edges:
        if uf.union(u, v):
            mst_weight += weight
            mst_edges.append((u, v, weight))
    
    return mst_weight, mst_edges

class UnionFind:
    def __init__(self, n):
        self.parent = list(range(n))
        self.rank = [0] * n
    
    def find(self, x):
        if self.parent[x] != x:
            self.parent[x] = self.find(self.parent[x])
        return self.parent[x]
    
    def union(self, x, y):
        px, py = self.find(x), self.find(y)
        if px == py:
            return False
        if self.rank[px] < self.rank[py]:
            px, py = py, px
        self.parent[py] = px
        if self.rank[px] == self.rank[py]:
            self.rank[px] += 1
        return True
\end{lstlisting}

\subsubsection{Prim's Algorithm}
\textbf{Description:} MST using heap. Good for dense graphs. Time: O(E log V).

\begin{lstlisting}
import heapq

def prim(graph, n):
    # graph[u] = [(v, weight), ...]
    visited = [False] * n
    min_heap = [(0, 0)]  # (weight, node)
    mst_weight = 0
    
    while min_heap:
        weight, u = heapq.heappop(min_heap)
        
        if visited[u]:
            continue
        
        visited[u] = True
        mst_weight += weight
        
        for v, w in graph[u]:
            if not visited[v]:
                heapq.heappush(min_heap, (w, v))
    
    return mst_weight
\end{lstlisting}

\newpage

\section{Topological Sort}

\textbf{Description:} Linear ordering of vertices in a DAG (Directed Acyclic Graph) such that for every edge u→v, u comes before v. Used for task scheduling, course prerequisites, build systems. Time: O(V+E).

\subsection{Kahn's Algorithm (BFS-based)}
\textbf{Advantages:} Detects cycles, can process nodes level by level.

\begin{lstlisting}
from collections import deque

def topo_sort(graph, n):
    # Count incoming edges for each node
    indegree = [0] * n
    for u in range(n):
        for v in graph[u]:
            indegree[v] += 1
    
    # Start with nodes having no dependencies
    queue = deque([i for i in range(n) 
                   if indegree[i] == 0])
    result = []
    
    while queue:
        node = queue.popleft()
        result.append(node)
        
        # Remove this node from graph
        for neighbor in graph[node]:
            indegree[neighbor] -= 1
            
            # If neighbor has no more dependencies
            if indegree[neighbor] == 0:
                queue.append(neighbor)
    
    # If not all nodes processed, cycle exists
    return result if len(result) == n else []
\end{lstlisting}

\subsection{DFS-based Topological Sort}
\textbf{Advantages:} Simpler code, uses less space.

\begin{lstlisting}
def topo_dfs(graph, n):
    visited = [False] * n
    stack = []
    
    def dfs(node):
        visited[node] = True
        
        # Visit all neighbors first
        for neighbor in graph[node]:
            if not visited[neighbor]:
                dfs(neighbor)
        
        # Add to stack after visiting all descendants
        stack.append(node)
    
    # Process all components
    for i in range(n):
        if not visited[i]:
            dfs(i)
    
    # Reverse stack gives topological order
    return stack[::-1]
\end{lstlisting}

\newpage

\section{Union-Find (Disjoint Set Union)}

\textbf{Description:} Efficiently tracks disjoint sets and supports union and find operations. Used for Kruskal's MST, connected components, cycle detection. Time: O($\alpha$(n)) $\approx$ O(1) per operation with path compression and union by rank.

\textbf{Applications:}
\begin{itemize}
\item Kruskal's minimum spanning tree
\item Detecting cycles in undirected graphs
\item Finding connected components
\item Network connectivity problems
\end{itemize}

\begin{lstlisting}
class UnionFind:
    def __init__(self, n):
        # Each node is its own parent initially
        self.parent = list(range(n))
        # Rank for union by rank optimization
        self.rank = [0] * n
    
    def find(self, x):
        # Path compression: point directly to root
        if self.parent[x] != x:
            self.parent[x] = self.find(self.parent[x])
        return self.parent[x]
    
    def union(self, x, y):
        # Find roots
        px, py = self.find(x), self.find(y)
        
        # Already in same set
        if px == py:
            return False
        
        # Union by rank: attach smaller tree under larger
        if self.rank[px] < self.rank[py]:
            px, py = py, px
        
        self.parent[py] = px
        
        # Increase rank if trees had equal rank
        if self.rank[px] == self.rank[py]:
            self.rank[px] += 1
        
        return True
    
    def connected(self, x, y):
        return self.find(x) == self.find(y)
    
    # Count number of disjoint sets
    def count_sets(self):
        return len(set(self.find(i) 
                   for i in range(len(self.parent))))

# Example: Detect cycle in undirected graph
def has_cycle_uf(edges, n):
    uf = UnionFind(n)
    for u, v in edges:
        if uf.connected(u, v):
            return True  # Cycle found
        uf.union(u, v)
    return False
\end{lstlisting}

\newpage

\section{Binary Search}

\textbf{Description:} Search in O(log n) time. Works on monotonic functions (sorted arrays, or functions where condition transitions from false to true exactly once).

\subsection{Template for Finding First/Last Position}

\begin{lstlisting}
# Find FIRST position where check(mid) is True
def binary_search_first(left, right, check):
    while left < right:
        mid = (left + right) // 2
        
        if check(mid):
            right = mid  # Could be answer, search left
        else:
            left = mid + 1  # Not answer, search right
    
    return left

# Find LAST position where check(mid) is True
def binary_search_last(left, right, check):
    while left < right:
        mid = (left + right + 1) // 2  # Round up!
        
        if check(mid):
            left = mid  # Could be answer, search right
        else:
            right = mid - 1  # Not answer, search left
    
    return left

# Example: Integer square root
def sqrt_binary(n):
    left, right = 0, n
    
    while left < right:
        mid = (left + right + 1) // 2
        
        if mid * mid <= n:
            left = mid  # mid might be answer
        else:
            right = mid - 1
    
    return left

# Binary search on answer - common pattern
def min_days_to_make_bouquets(bloomDay, m, k):
    # Can we make m bouquets in 'days' days?
    def can_make(days):
        bouquets = consecutive = 0
        for bloom in bloomDay:
            if bloom <= days:
                consecutive += 1
                if consecutive == k:
                    bouquets += 1
                    consecutive = 0
            else:
                consecutive = 0
        return bouquets >= m
    
    if len(bloomDay) < m * k:
        return -1
    
    # Binary search on number of days
    return binary_search_first(
        min(bloomDay), max(bloomDay), can_make)
\end{lstlisting}

\newpage

\section{Dynamic Programming}

\textbf{Description:} Solve problems by breaking them into overlapping subproblems. Store results to avoid recomputation.

\subsection{Longest Increasing Subsequence}
\textbf{Description:} Find length of longest strictly increasing subsequence. Time: O(n log n) using binary search.

\begin{lstlisting}
def lis(arr):
    from bisect import bisect_left
    
    # dp[i] = smallest ending value of LIS of length i+1
    dp = []
    
    for x in arr:
        # Find position to place x
        idx = bisect_left(dp, x)
        
        if idx == len(dp):
            dp.append(x)  # Extend LIS
        else:
            dp[idx] = x   # Better ending for this length
    
    return len(dp)

# LIS with actual sequence
def lis_with_sequence(arr):
    from bisect import bisect_left
    
    n = len(arr)
    dp = []
    parent = [-1] * n
    dp_idx = []  # indices in dp
    
    for i, x in enumerate(arr):
        idx = bisect_left(dp, x)
        
        if idx == len(dp):
            dp.append(x)
            dp_idx.append(i)
        else:
            dp[idx] = x
            dp_idx[idx] = i
        
        if idx > 0:
            parent[i] = dp_idx[idx - 1]
    
    # Reconstruct sequence
    result = []
    idx = dp_idx[-1]
    while idx != -1:
        result.append(arr[idx])
        idx = parent[idx]
    
    return result[::-1]
\end{lstlisting}

\subsection{0/1 Knapsack}
\textbf{Description:} Maximum value with weight capacity. Each item can be taken 0 or 1 time. Time: O(n×capacity), Space: O(n×capacity).

\begin{lstlisting}
def knapsack(weights, values, capacity):
    n = len(weights)
    # dp[i][w] = max value using first i items, 
    #            weight <= w
    dp = [[0] * (capacity + 1) for _ in range(n + 1)]
    
    for i in range(1, n + 1):
        for w in range(capacity + 1):
            # Don't take item i-1
            dp[i][w] = dp[i-1][w]
            
            # Take item i-1 if it fits
            if weights[i-1] <= w:
                dp[i][w] = max(
                    dp[i][w],
                    dp[i-1][w - weights[i-1]] + values[i-1]
                )
    
    return dp[n][capacity]

# Space-optimized O(capacity)
def knapsack_optimized(weights, values, capacity):
    dp = [0] * (capacity + 1)
    
    for i in range(len(weights)):
        # Iterate backwards to avoid using updated values
        for w in range(capacity, weights[i] - 1, -1):
            dp[w] = max(dp[w], 
                       dp[w - weights[i]] + values[i])
    
    return dp[capacity]
\end{lstlisting}

\subsection{Edit Distance (Levenshtein Distance)}
\textbf{Description:} Minimum operations (insert, delete, replace) to transform s1 to s2. Time: O(m×n), Space: O(m×n).

\begin{lstlisting}
def edit_dist(s1, s2):
    m, n = len(s1), len(s2)
    # dp[i][j] = edit distance of s1[:i] and s2[:j]
    dp = [[0] * (n + 1) for _ in range(m + 1)]
    
    # Base cases: empty string transformations
    for i in range(m + 1):
        dp[i][0] = i  # Delete all
    for j in range(n + 1):
        dp[0][j] = j  # Insert all
    
    for i in range(1, m + 1):
        for j in range(1, n + 1):
            if s1[i-1] == s2[j-1]:
                # Characters match, no operation needed
                dp[i][j] = dp[i-1][j-1]
            else:
                dp[i][j] = 1 + min(
                    dp[i-1][j],      # Delete from s1
                    dp[i][j-1],      # Insert into s1
                    dp[i-1][j-1]     # Replace in s1
                )
    
    return dp[m][n]
\end{lstlisting}

\subsection{Longest Common Subsequence (LCS)}
\textbf{Description:} Longest subsequence common to two sequences. Time: O(m×n).

\begin{lstlisting}
def lcs(s1, s2):
    m, n = len(s1), len(s2)
    dp = [[0] * (n + 1) for _ in range(m + 1)]
    
    for i in range(1, m + 1):
        for j in range(1, n + 1):
            if s1[i-1] == s2[j-1]:
                dp[i][j] = dp[i-1][j-1] + 1
            else:
                dp[i][j] = max(dp[i-1][j], dp[i][j-1])
    
    return dp[m][n]

# Reconstruct LCS
def lcs_string(s1, s2):
    m, n = len(s1), len(s2)
    dp = [[0] * (n + 1) for _ in range(m + 1)]
    
    for i in range(1, m + 1):
        for j in range(1, n + 1):
            if s1[i-1] == s2[j-1]:
                dp[i][j] = dp[i-1][j-1] + 1
            else:
                dp[i][j] = max(dp[i-1][j], dp[i][j-1])
    
    # Backtrack
    result = []
    i, j = m, n
    while i > 0 and j > 0:
        if s1[i-1] == s2[j-1]:
            result.append(s1[i-1])
            i -= 1
            j -= 1
        elif dp[i-1][j] > dp[i][j-1]:
            i -= 1
        else:
            j -= 1
    
    return ''.join(reversed(result))
\end{lstlisting}

\subsection{Coin Change}
\textbf{Description:} Minimum coins to make amount, or count ways. Time: O(n×amount).

\begin{lstlisting}
# Minimum coins
def coin_change_min(coins, amount):
    dp = [float('inf')] * (amount + 1)
    dp[0] = 0
    
    for coin in coins:
        for i in range(coin, amount + 1):
            dp[i] = min(dp[i], dp[i - coin] + 1)
    
    return dp[amount] if dp[amount] != float('inf') else -1

# Count ways
def coin_change_ways(coins, amount):
    dp = [0] * (amount + 1)
    dp[0] = 1
    
    for coin in coins:
        for i in range(coin, amount + 1):
            dp[i] += dp[i - coin]
    
    return dp[amount]
\end{lstlisting}

\subsection{Palindrome Partitioning}
\textbf{Description:} Minimum cuts to partition string into palindromes. Time: O(n²).

\begin{lstlisting}
def min_palindrome_partition(s):
    n = len(s)
    
    # is_pal[i][j] = True if s[i:j+1] is palindrome
    is_pal = [[False] * n for _ in range(n)]
    
    # Every single character is palindrome
    for i in range(n):
        is_pal[i][i] = True
    
    # Check all substrings
    for length in range(2, n + 1):
        for i in range(n - length + 1):
            j = i + length - 1
            if s[i] == s[j]:
                is_pal[i][j] = (length == 2 or 
                               is_pal[i+1][j-1])
    
    # dp[i] = min cuts for s[0:i+1]
    dp = [float('inf')] * n
    
    for i in range(n):
        if is_pal[0][i]:
            dp[i] = 0
        else:
            for j in range(i):
                if is_pal[j+1][i]:
                    dp[i] = min(dp[i], dp[j] + 1)
    
    return dp[n-1]
\end{lstlisting}

\subsection{Subset Sum}
\textbf{Description:} Check if subset sums to target. Time: O(n×sum).

\begin{lstlisting}
def subset_sum(arr, target):
    n = len(arr)
    dp = [[False] * (target + 1) for _ in range(n + 1)]
    
    # Base case: sum 0 is always achievable
    for i in range(n + 1):
        dp[i][0] = True
    
    for i in range(1, n + 1):
        for s in range(target + 1):
            # Don't take arr[i-1]
            dp[i][s] = dp[i-1][s]
            
            # Take arr[i-1] if possible
            if s >= arr[i-1]:
                dp[i][s] = dp[i][s] or dp[i-1][s - arr[i-1]]
    
    return dp[n][target]

# Space optimized
def subset_sum_optimized(arr, target):
    dp = [False] * (target + 1)
    dp[0] = True
    
    for num in arr:
        for s in range(target, num - 1, -1):
            dp[s] = dp[s] or dp[s - num]
    
    return dp[target]
\end{lstlisting}

\newpage

\section{Array Techniques}

\subsection{Prefix Sum}
\textbf{Description:} Precompute cumulative sums for O(1) range queries. Time: O(n) preprocessing, O(1) query.

\begin{lstlisting}
# 1D prefix sum
prefix = [0] * (n + 1)
for i in range(n):
    prefix[i + 1] = prefix[i] + arr[i]

# Range sum query [l, r] inclusive
range_sum = prefix[r + 1] - prefix[l]

# 2D prefix sum - for rectangle sum queries
def build_2d_prefix(matrix):
    n, m = len(matrix), len(matrix[0])
    prefix = [[0] * (m + 1) for _ in range(n + 1)]
    
    for i in range(1, n + 1):
        for j in range(1, m + 1):
            prefix[i][j] = (matrix[i-1][j-1] + 
                           prefix[i-1][j] + 
                           prefix[i][j-1] - 
                           prefix[i-1][j-1])
    
    return prefix

# Rectangle sum from (x1,y1) to (x2,y2) inclusive
def rect_sum(prefix, x1, y1, x2, y2):
    return (prefix[x2+1][y2+1] - 
            prefix[x1][y2+1] - 
            prefix[x2+1][y1] + 
            prefix[x1][y1])
\end{lstlisting}

\subsection{Difference Array}
\textbf{Description:} Efficiently perform range updates. O(1) per update, O(n) to reconstruct.

\begin{lstlisting}
# Initialize difference array
diff = [0] * (n + 1)

# Add 'val' to range [l, r]
def range_update(diff, l, r, val):
    diff[l] += val
    diff[r + 1] -= val

# After all updates, reconstruct array
def reconstruct(diff):
    result = []
    current = 0
    for i in range(len(diff) - 1):
        current += diff[i]
        result.append(current)
    return result

# Example: Multiple range updates
diff = [0] * (n + 1)
for l, r, val in updates:
    range_update(diff, l, r, val)
final_array = reconstruct(diff)
\end{lstlisting}

\subsection{Sliding Window}
\textbf{Description:} Maintain a window of elements while traversing. Time: O(n).

\begin{lstlisting}
# Fixed size window
def max_sum_window(arr, k):
    window_sum = sum(arr[:k])
    max_sum = window_sum
    
    # Slide window: add right, remove left
    for i in range(k, len(arr)):
        window_sum += arr[i] - arr[i - k]
        max_sum = max(max_sum, window_sum)
    
    return max_sum

# Variable size window - two pointers
def min_subarray_sum_geq_target(arr, target):
    left = 0
    current_sum = 0
    min_len = float('inf')
    
    for right in range(len(arr)):
        current_sum += arr[right]
        
        # Shrink window while condition holds
        while current_sum >= target:
            min_len = min(min_len, right - left + 1)
            current_sum -= arr[left]
            left += 1
    
    return min_len if min_len != float('inf') else 0

# Longest substring with at most k distinct chars
def longest_k_distinct(s, k):
    from collections import defaultdict
    
    left = 0
    char_count = defaultdict(int)
    max_len = 0
    
    for right in range(len(s)):
        char_count[s[right]] += 1
        
        # Shrink if too many distinct
        while len(char_count) > k:
            char_count[s[left]] -= 1
            if char_count[s[left]] == 0:
                del char_count[s[left]]
            left += 1
        
        max_len = max(max_len, right - left + 1)
    
    return max_len
\end{lstlisting}

\newpage

\section{Advanced Data Structures}

\subsection{Segment Tree}
\textbf{Description:} Supports range queries and point updates in O(log n). Can be modified for range updates with lazy propagation.

\begin{lstlisting}
class SegmentTree:
    def __init__(self, arr):
        self.n = len(arr)
        # Tree size: 4n is safe upper bound
        self.tree = [0] * (4 * self.n)
        self.build(arr, 0, 0, self.n - 1)
    
    def build(self, arr, node, start, end):
        if start == end:
            # Leaf node
            self.tree[node] = arr[start]
        else:
            mid = (start + end) // 2
            # Build left and right subtrees
            self.build(arr, 2*node+1, start, mid)
            self.build(arr, 2*node+2, mid+1, end)
            # Combine results (sum in this case)
            self.tree[node] = (self.tree[2*node+1] + 
                              self.tree[2*node+2])
    
    def update(self, node, start, end, idx, val):
        if start == end:
            # Leaf node - update value
            self.tree[node] = val
        else:
            mid = (start + end) // 2
            if idx <= mid:
                # Update left subtree
                self.update(2*node+1, start, mid, idx, val)
            else:
                # Update right subtree
                self.update(2*node+2, mid+1, end, idx, val)
            # Recompute parent
            self.tree[node] = (self.tree[2*node+1] + 
                              self.tree[2*node+2])
    
    def query(self, node, start, end, l, r):
        # No overlap
        if r < start or end < l:
            return 0
        
        # Complete overlap
        if l <= start and end <= r:
            return self.tree[node]
        
        # Partial overlap
        mid = (start + end) // 2
        left_sum = self.query(2*node+1, start, mid, l, r)
        right_sum = self.query(2*node+2, mid+1, end, l, r)
        return left_sum + right_sum
    
    # Public interface
    def update_val(self, idx, val):
        self.update(0, 0, self.n-1, idx, val)
    
    def range_sum(self, l, r):
        return self.query(0, 0, self.n-1, l, r)
\end{lstlisting}

\subsection{Fenwick Tree (Binary Indexed Tree)}
\textbf{Description:} Simpler than segment tree, supports prefix sum and point updates in O(log n). More space efficient.

\begin{lstlisting}
class FenwickTree:
    def __init__(self, n):
        self.n = n
        # 1-indexed for easier implementation
        self.tree = [0] * (n + 1)
    
    def update(self, i, delta):
        # Add delta to position i (1-indexed)
        while i <= self.n:
            self.tree[i] += delta
            # Move to next node: add LSB
            i += i & (-i)
    
    def query(self, i):
        # Get prefix sum up to i (1-indexed)
        s = 0
        while i > 0:
            s += self.tree[i]
            # Move to parent: remove LSB
            i -= i & (-i)
        return s
    
    def range_query(self, l, r):
        # Sum from l to r (1-indexed)
        return self.query(r) - self.query(l - 1)

# Usage example
bit = FenwickTree(n)
for i, val in enumerate(arr, 1):
    bit.update(i, val)

# Range sum [l, r] (1-indexed)
result = bit.range_query(l, r)
\end{lstlisting}

\subsection{Trie (Prefix Tree)}
\textbf{Description:} Tree for storing strings, enables fast prefix searches. Time: O(m) for operations where m is string length.

\begin{lstlisting}
class TrieNode:
    def __init__(self):
        self.children = {}  # char -> TrieNode
        self.is_end = False  # End of word marker

class Trie:
    def __init__(self):
        self.root = TrieNode()
    
    def insert(self, word):
        # Insert word - O(len(word))
        node = self.root
        for char in word:
            if char not in node.children:
                node.children[char] = TrieNode()
            node = node.children[char]
        node.is_end = True
    
    def search(self, word):
        # Exact word search - O(len(word))
        node = self.root
        for char in word:
            if char not in node.children:
                return False
            node = node.children[char]
        return node.is_end
    
    def starts_with(self, prefix):
        # Prefix search - O(len(prefix))
        node = self.root
        for char in prefix:
            if char not in node.children:
                return False
            node = node.children[char]
        return True
    
    # Find all words with given prefix
    def words_with_prefix(self, prefix):
        node = self.root
        for char in prefix:
            if char not in node.children:
                return []
            node = node.children[char]
        
        # DFS to collect all words
        words = []
        def dfs(n, path):
            if n.is_end:
                words.append(prefix + path)
            for char, child in n.children.items():
                dfs(child, path + char)
        
        dfs(node, "")
        return words
\end{lstlisting}

\newpage

\section{Bit Manipulation}

\textbf{Description:} Efficient operations using bitwise operators. Useful for sets, flags, and optimization.

\begin{lstlisting}
# Check if i-th bit (0-indexed) is set
is_set = (n >> i) & 1

# Set i-th bit to 1
n |= (1 << i)

# Clear i-th bit (set to 0)
n &= ~(1 << i)

# Toggle i-th bit
n ^= (1 << i)

# Count set bits (popcount)
count = bin(n).count('1')
count = n.bit_count()  # Python 3.10+

# Get lowest set bit
lsb = n & -n  # Also n & (~n + 1)

# Remove lowest set bit
n &= (n - 1)

# Check if power of 2
is_pow2 = n > 0 and (n & (n - 1)) == 0

# Check if power of 4
is_pow4 = n > 0 and (n & (n-1)) == 0 and (n & 0x55555555) != 0

# Iterate over all subsets of set represented by mask
mask = (1 << n) - 1  # All bits set
submask = mask
while submask > 0:
    # Process submask
    submask = (submask - 1) & mask

# Iterate through all k-bit masks
def iterate_k_bits(n, k):
    mask = (1 << k) - 1
    while mask < (1 << n):
        # Process mask
        yield mask
        # Gosper's hack
        c = mask & -mask
        r = mask + c
        mask = (((r ^ mask) >> 2) // c) | r

# XOR properties
# a ^ a = 0 (number XOR itself is 0)
# a ^ 0 = a (number XOR 0 is itself)
# XOR is commutative and associative
# Find unique element when all others appear twice:
def find_unique(arr):
    result = 0
    for x in arr:
        result ^= x
    return result

# Subset enumeration
n = 5  # Number of elements
for mask in range(1 << n):
    subset = [i for i in range(n) if mask & (1 << i)]
    # Process subset

# Check parity (odd/even number of 1s)
def parity(n):
    count = 0
    while n:
        count ^= 1
        n &= n - 1
    return count  # 1 if odd, 0 if even

# Swap two numbers without temp variable
a, b = 5, 10
a ^= b
b ^= a
a ^= b
# Now a=10, b=5
\end{lstlisting}

\newpage

\section{Matrix Operations}

\textbf{Description:} Matrix operations for DP optimization, graph algorithms, and recurrence relations.

\subsection{Matrix Multiplication}
\begin{lstlisting}
# Standard matrix multiplication - O(n^3)
def matmul(A, B):
    n, m, p = len(A), len(A[0]), len(B[0])
    C = [[0] * p for _ in range(n)]
    
    for i in range(n):
        for j in range(p):
            for k in range(m):
                C[i][j] += A[i][k] * B[k][j]
    
    return C

# With modulo
def matmul_mod(A, B, mod):
    n = len(A)
    C = [[0] * n for _ in range(n)]
    
    for i in range(n):
        for j in range(n):
            for k in range(n):
                C[i][j] = (C[i][j] + 
                          A[i][k] * B[k][j]) % mod
    
    return C
\end{lstlisting}

\subsection{Matrix Exponentiation}
\textbf{Description:} Compute M\textsuperscript{n} in O(k\textsuperscript{3} log n) where k is matrix dimension. Used for solving linear recurrences efficiently.

\begin{lstlisting}
def matpow(M, n, mod):
    size = len(M)
    
    # Identity matrix
    result = [[1 if i==j else 0 
               for j in range(size)] 
              for i in range(size)]
    
    # Binary exponentiation
    while n > 0:
        if n & 1:
            result = matmul_mod(result, M, mod)
        M = matmul_mod(M, M, mod)
        n >>= 1
    
    return result

# Example: Fibonacci using matrix exponentiation
# F(n) = [[1,1],[1,0]]^n
def fibonacci(n, mod):
    if n == 0: return 0
    if n == 1: return 1
    
    M = [[1, 1], [1, 0]]
    result = matpow(M, n - 1, mod)
    return result[0][0]

# Linear recurrence: a(n) = c1*a(n-1) + c2*a(n-2) + ...
# Build transition matrix and use matrix exponentiation
def linear_recurrence(coeffs, init, n, mod):
    k = len(coeffs)
    
    # Transition matrix
    # [a(n), a(n-1), ..., a(n-k+1)]
    M = [[0] * k for _ in range(k)]
    M[0] = coeffs  # First row
    for i in range(1, k):
        M[i][i-1] = 1  # Identity for shifting
    
    # Initial state vector
    state = init[::-1]  # Reverse order
    
    if n < k:
        return init[n]
    
    # M^(n-k+1)
    result_matrix = matpow(M, n - k + 1, mod)
    
    # Multiply with initial state
    result = 0
    for i in range(k):
        result = (result + result_matrix[0][i] * state[i]) % mod
    
    return result
\end{lstlisting}

\newpage

\section{Miscellaneous Tips}

\subsection{Python-Specific Optimizations}
\begin{lstlisting}
# Fast input for large datasets
import sys
input = sys.stdin.readline

# Increase recursion limit for deep DFS/DP
sys.setrecursionlimit(10**6)

# Deep copy (be careful with performance)
from copy import deepcopy
new_list = deepcopy(old_list)
\end{lstlisting}

\subsection{Useful Libraries}
\begin{lstlisting}
# Iterator tools - powerful combinations
from itertools import *

# permutations(iterable, r) - all r-length permutations
perms = list(permutations([1,2,3], 2))
# [(1,2), (1,3), (2,1), (2,3), (3,1), (3,2)]

# combinations(iterable, r) - r-length combinations
combs = list(combinations([1,2,3], 2))
# [(1,2), (1,3), (2,3)]

# product - cartesian product
prod = list(product([1,2], ['a','b']))
# [(1,'a'), (1,'b'), (2,'a'), (2,'b')]

# accumulate - running totals
acc = list(accumulate([1,2,3,4]))
# [1, 3, 6, 10]

# chain - flatten iterables
chained = list(chain([1,2], [3,4]))
# [1, 2, 3, 4]
\end{lstlisting}

\subsection{Common Patterns}
\begin{lstlisting}
# Lambda sorting with multiple keys
arr.sort(key=lambda x: (-x[0], x[1]))
# Sort by first desc, then second asc

# All/Any - short-circuit evaluation
all(x > 0 for x in arr)  # True if all positive
any(x > 0 for x in arr)  # True if any positive

# Zip - parallel iteration
for a, b in zip(list1, list2):
    pass

# Enumerate - index and value
for i, val in enumerate(arr):
    print(f"arr[{i}] = {val}")

# Custom comparison function
from functools import cmp_to_key

def compare(a, b):
    # Return -1 if a < b, 0 if equal, 1 if a > b
    if a + b > b + a:
        return -1
    return 1

arr.sort(key=cmp_to_key(compare))

# DefaultDict with lambda
from collections import defaultdict
d = defaultdict(lambda: float('inf'))

# Multiple assignment
a, b = b, a  # Swap
a, *rest, b = [1,2,3,4,5]  # a=1, rest=[2,3,4], b=5
\end{lstlisting}

\subsection{Common Pitfalls}
\begin{lstlisting}
# Integer division - floors toward negative infinity
print(7 // 3)    # 2
print(-7 // 3)   # -3 (not -2!)

# For ceiling division toward zero:
def div_ceil(a, b):
    return -(-a // b)

# Modulo with negative numbers
print((-5) % 3)  # 1 (not -2!)
print(5 % -3)    # -1

# List multiplication creates references!
matrix = [[0] * m] * n  # WRONG! All rows same object
matrix[0][0] = 1        # Changes all rows!

# Correct way
matrix = [[0] * m for _ in range(n)]

# Float comparison - don't use ==
a, b = 0.1 + 0.2, 0.3
print(a == b)  # False!

# Use epsilon comparison
eps = 1e-9
print(abs(a - b) < eps)  # True

# String immutability
s = "abc"
# s[0] = 'd'  # ERROR!
s = 'd' + s[1:]  # OK

# For many string mutations, use list
chars = list(s)
chars[0] = 'd'
s = ''.join(chars)

# Mutable default arguments - dangerous!
def func(arr=[]):  # WRONG!
    arr.append(1)
    return arr

# Each call modifies same list
print(func())  # [1]
print(func())  # [1, 1]

# Correct way
def func(arr=None):
    if arr is None:
        arr = []
    arr.append(1)
    return arr
\end{lstlisting}

\newpage

\section{Computational Geometry}

\subsection{Basic Geometry}
\textbf{Description:} Fundamental geometric operations for 2D points.

\begin{lstlisting}
import math

# Point operations
def dist(p1, p2):
    # Euclidean distance
    return math.sqrt((p1[0] - p2[0])**2 + (p1[1] - p2[1])**2)

def cross_product(O, A, B):
    # Cross product of vectors OA and OB
    # Positive: counter-clockwise
    # Negative: clockwise
    # Zero: collinear
    return (A[0] - O[0]) * (B[1] - O[1]) - \
           (A[1] - O[1]) * (B[0] - O[0])

def dot_product(A, B, C, D):
    # Dot product of vectors AB and CD
    return (B[0] - A[0]) * (D[0] - C[0]) + \
           (B[1] - A[1]) * (D[1] - C[1])

# Check if point is on segment
def on_segment(p, q, r):
    # Check if q lies on segment pr
    return (q[0] <= max(p[0], r[0]) and 
            q[0] >= min(p[0], r[0]) and
            q[1] <= max(p[1], r[1]) and 
            q[1] >= min(p[1], r[1]))

# Segment intersection
def segments_intersect(p1, q1, p2, q2):
    o1 = cross_product(p1, q1, p2)
    o2 = cross_product(p1, q1, q2)
    o3 = cross_product(p2, q2, p1)
    o4 = cross_product(p2, q2, q1)
    
    # General case
    if o1 * o2 < 0 and o3 * o4 < 0:
        return True
    
    # Special cases (collinear)
    if o1 == 0 and on_segment(p1, p2, q1):
        return True
    if o2 == 0 and on_segment(p1, q2, q1):
        return True
    if o3 == 0 and on_segment(p2, p1, q2):
        return True
    if o4 == 0 and on_segment(p2, q1, q2):
        return True
    
    return False
\end{lstlisting}

\subsection{Convex Hull}
\textbf{Description:} Find convex hull using Graham's scan. Time: O(n log n).

\begin{lstlisting}
def convex_hull(points):
    # Graham's scan algorithm
    points = sorted(points)  # Sort by x, then y
    
    if len(points) <= 2:
        return points
    
    # Build lower hull
    lower = []
    for p in points:
        while (len(lower) >= 2 and 
               cross_product(lower[-2], lower[-1], p) <= 0):
            lower.pop()
        lower.append(p)
    
    # Build upper hull
    upper = []
    for p in reversed(points):
        while (len(upper) >= 2 and 
               cross_product(upper[-2], upper[-1], p) <= 0):
            upper.pop()
        upper.append(p)
    
    # Remove last point (duplicate of first)
    return lower[:-1] + upper[:-1]

# Convex hull area
def polygon_area(points):
    # Shoelace formula
    n = len(points)
    area = 0
    
    for i in range(n):
        j = (i + 1) % n
        area += points[i][0] * points[j][1]
        area -= points[j][0] * points[i][1]
    
    return abs(area) / 2
\end{lstlisting}

\subsection{Point in Polygon}
\textbf{Description:} Check if point is inside polygon. Time: O(n).

\begin{lstlisting}
def point_in_polygon(point, polygon):
    # Ray casting algorithm
    x, y = point
    n = len(polygon)
    inside = False
    
    p1x, p1y = polygon[0]
    for i in range(1, n + 1):
        p2x, p2y = polygon[i % n]
        
        if y > min(p1y, p2y):
            if y <= max(p1y, p2y):
                if x <= max(p1x, p2x):
                    if p1y != p2y:
                        xinters = (y - p1y) * (p2x - p1x) / \
                                  (p2y - p1y) + p1x
                    
                    if p1x == p2x or x <= xinters:
                        inside = not inside
        
        p1x, p1y = p2x, p2y
    
    return inside
\end{lstlisting}

\subsection{Closest Pair of Points}
\textbf{Description:} Find closest pair using divide and conquer. Time: O(n log n).

\begin{lstlisting}
def closest_pair(points):
    points_sorted_x = sorted(points, key=lambda p: p[0])
    points_sorted_y = sorted(points, key=lambda p: p[1])
    
    def closest_recursive(px, py):
        n = len(px)
        
        # Base case: brute force
        if n <= 3:
            min_dist = float('inf')
            for i in range(n):
                for j in range(i + 1, n):
                    min_dist = min(min_dist, dist(px[i], px[j]))
            return min_dist
        
        # Divide
        mid = n // 2
        midpoint = px[mid]
        
        pyl = [p for p in py if p[0] <= midpoint[0]]
        pyr = [p for p in py if p[0] > midpoint[0]]
        
        # Conquer
        dl = closest_recursive(px[:mid], pyl)
        dr = closest_recursive(px[mid:], pyr)
        d = min(dl, dr)
        
        # Combine: check strip
        strip = [p for p in py if abs(p[0] - midpoint[0]) < d]
        
        for i in range(len(strip)):
            j = i + 1
            while j < len(strip) and strip[j][1] - strip[i][1] < d:
                d = min(d, dist(strip[i], strip[j]))
                j += 1
        
        return d
    
    return closest_recursive(points_sorted_x, points_sorted_y)
\end{lstlisting}

\newpage

\section{Network Flow}

\subsection{Maximum Flow - Edmonds-Karp (BFS-based Ford-Fulkerson)}
\textbf{Description:} Find maximum flow from source to sink. Time: O(VE²).

\begin{lstlisting}
from collections import deque, defaultdict

def max_flow(graph, source, sink, n):
    # graph[u][v] = capacity from u to v
    # Build residual graph
    residual = defaultdict(lambda: defaultdict(int))
    for u in graph:
        for v in graph[u]:
            residual[u][v] = graph[u][v]
    
    def bfs_path():
        # Find augmenting path using BFS
        parent = {source: None}
        visited = {source}
        queue = deque([source])
        
        while queue:
            u = queue.popleft()
            
            if u == sink:
                # Reconstruct path
                path = []
                while parent[u] is not None:
                    path.append((parent[u], u))
                    u = parent[u]
                return path[::-1]
            
            for v in range(n):
                if v not in visited and residual[u][v] > 0:
                    visited.add(v)
                    parent[v] = u
                    queue.append(v)
        
        return None
    
    max_flow_value = 0
    
    # Find augmenting paths
    while True:
        path = bfs_path()
        if path is None:
            break
        
        # Find minimum capacity along path
        flow = min(residual[u][v] for u, v in path)
        
        # Update residual graph
        for u, v in path:
            residual[u][v] -= flow
            residual[v][u] += flow
        
        max_flow_value += flow
    
    return max_flow_value

# Example usage
# graph[u][v] = capacity
graph = defaultdict(lambda: defaultdict(int))
graph[0][1] = 10
graph[0][2] = 10
graph[1][3] = 4
graph[1][4] = 8
graph[2][4] = 9
graph[3][5] = 10
graph[4][3] = 6
graph[4][5] = 10

n = 6  # Number of nodes
result = max_flow(graph, 0, 5, n)
\end{lstlisting}

\subsection{Dinic's Algorithm (Faster)}
\textbf{Description:} Faster max flow using level graph and blocking flow. Time: O(V²E).

\begin{lstlisting}
from collections import deque, defaultdict

class Dinic:
    def __init__(self, n):
        self.n = n
        self.graph = defaultdict(lambda: defaultdict(int))
    
    def add_edge(self, u, v, cap):
        self.graph[u][v] += cap
    
    def bfs(self, source, sink):
        # Build level graph
        level = [-1] * self.n
        level[source] = 0
        queue = deque([source])
        
        while queue:
            u = queue.popleft()
            
            for v in range(self.n):
                if level[v] == -1 and self.graph[u][v] > 0:
                    level[v] = level[u] + 1
                    queue.append(v)
        
        return level if level[sink] != -1 else None
    
    def dfs(self, u, sink, pushed, level, start):
        if u == sink:
            return pushed
        
        while start[u] < self.n:
            v = start[u]
            
            if (level[v] == level[u] + 1 and 
                self.graph[u][v] > 0):
                
                flow = self.dfs(v, sink, 
                               min(pushed, self.graph[u][v]),
                               level, start)
                
                if flow > 0:
                    self.graph[u][v] -= flow
                    self.graph[v][u] += flow
                    return flow
            
            start[u] += 1
        
        return 0
    
    def max_flow(self, source, sink):
        flow = 0
        
        while True:
            level = self.bfs(source, sink)
            if level is None:
                break
            
            start = [0] * self.n
            
            while True:
                pushed = self.dfs(source, sink, float('inf'),
                                 level, start)
                if pushed == 0:
                    break
                flow += pushed
        
        return flow
\end{lstlisting}

\subsection{Min Cut}
\textbf{Description:} Find minimum cut after computing max flow.

\begin{lstlisting}
def min_cut(graph, source, n, residual):
    # After running max_flow, residual graph is available
    # Min cut = set of reachable nodes from source
    visited = [False] * n
    queue = deque([source])
    visited[source] = True
    
    while queue:
        u = queue.popleft()
        for v in range(n):
            if not visited[v] and residual[u][v] > 0:
                visited[v] = True
                queue.append(v)
    
    # Cut edges
    cut_edges = []
    for u in range(n):
        if visited[u]:
            for v in range(n):
                if not visited[v] and graph[u][v] > 0:
                    cut_edges.append((u, v))
    
    return cut_edges
\end{lstlisting}

\subsection{Bipartite Matching}
\textbf{Description:} Maximum matching in bipartite graph using flow.

\begin{lstlisting}
def max_bipartite_matching(left_size, right_size, edges):
    # edges = [(left_node, right_node), ...]
    # Add source (0) and sink (left_size + right_size + 1)
    
    n = left_size + right_size + 2
    source = 0
    sink = n - 1
    
    graph = defaultdict(lambda: defaultdict(int))
    
    # Source to left nodes
    for i in range(1, left_size + 1):
        graph[source][i] = 1
    
    # Left to right edges
    for l, r in edges:
        graph[l + 1][left_size + r + 1] = 1
    
    # Right nodes to sink
    for i in range(1, right_size + 1):
        graph[left_size + i][sink] = 1
    
    return max_flow(graph, source, sink, n)
\end{lstlisting}


\end{document}
\begin{lstlisting}
# Fast I/O
import sys
input = sys.stdin.readline

# Read integers
n = int(input())
a, b = map(int, input().split())
arr = list(map(int, input().split()))

# Read strings
s = input().strip()
words = input().split()

# Multiple test cases
t = int(input())
for _ in range(t):
    # process

# Print without newline
print(x, end=' ')

# Formatted output
print(f"{x:.6f}")  # 6 decimals
\end{lstlisting}

\subsection*{List Operations}
\begin{lstlisting}
# Initialize
arr = [0] * n
matrix = [[0] * m for _ in range(n)]

# List comprehension
squares = [x**2 for x in range(n)]
evens = [x for x in arr if x % 2 == 0]

# Sorting
arr.sort()  # in-place
arr.sort(reverse=True)
arr.sort(key=lambda x: (x[0], -x[1]))
sorted_arr = sorted(arr)

# Binary search
from bisect import bisect_left, bisect_right
idx = bisect_left(arr, x)  # leftmost
idx = bisect_right(arr, x)  # rightmost

# Useful operations
arr.append(x)
arr.pop()  # last element
arr.pop(0)  # first (O(n))
arr.reverse()
arr.count(x)
arr.index(x)  # first occurrence
\end{lstlisting}

\subsection*{Deque (Double-ended Queue)}
\begin{lstlisting}
from collections import deque
dq = deque()

dq.append(x)      # add to right O(1)
dq.appendleft(x)  # add to left O(1)
dq.pop()          # remove from right O(1)
dq.popleft()      # remove from left O(1)

# Sliding window maximum
def sliding_max(arr, k):
    dq = deque()
    result = []
    for i in range(len(arr)):
        # Remove out of window
        while dq and dq[0] < i - k + 1:
            dq.popleft()
        # Remove smaller elements
        while dq and arr[dq[-1]] < arr[i]:
            dq.pop()
        dq.append(i)
        if i >= k - 1:
            result.append(arr[dq[0]])
    return result
\end{lstlisting}

\subsection*{Heap (Priority Queue)}
\begin{lstlisting}
import heapq

# Min heap (default)
heap = []
heapq.heappush(heap, x)
min_val = heapq.heappop(heap)
min_val = heap[0]  # peek

# Max heap (negate values)
heapq.heappush(heap, -x)
max_val = -heapq.heappop(heap)

# Heapify existing list
heapq.heapify(arr)

# K largest/smallest
k_largest = heapq.nlargest(k, arr)
k_smallest = heapq.nsmallest(k, arr)

# Custom comparator (use tuples)
heapq.heappush(heap, (priority, item))
\end{lstlisting}

\subsection*{Dictionary \& Counter}
\begin{lstlisting}
from collections import defaultdict, Counter

# defaultdict
graph = defaultdict(list)
count = defaultdict(int)

# Counter
cnt = Counter(arr)
cnt['x'] += 1
most_common = cnt.most_common(k)

# Dictionary operations
d = {}
d.get(key, default_val)
d.setdefault(key, default_val)
for k, v in d.items():
    pass
\end{lstlisting}

\subsection*{Set Operations}
\begin{lstlisting}
s = set()
s.add(x)
s.remove(x)  # KeyError if not exists
s.discard(x)  # no error

# Set operations
a | b  # union
a & b  # intersection
a - b  # difference
a ^ b  # symmetric difference

# Ordered set (using dict)
from collections import OrderedDict
oset = OrderedDict.fromkeys([])
\end{lstlisting}

\subsection*{String Operations}
\begin{lstlisting}
# Common methods
s.lower(), s.upper()
s.strip(), s.lstrip(), s.rstrip()
s.split(delimiter)
delimiter.join(list)
s.replace(old, new)
s.startswith(prefix)
s.endswith(suffix)
s.isdigit(), s.isalpha(), s.isalnum()

# String building (efficient)
result = []
for x in data:
    result.append(str(x))
s = ''.join(result)

# ASCII values
ord('a')  # 97
chr(97)   # 'a'
\end{lstlisting}

\subsection*{Math Operations}
\begin{lstlisting}
import math

# Common functions
math.ceil(x), math.floor(x)
math.gcd(a, b)
math.lcm(a, b)  # Python 3.9+
math.sqrt(x)
math.log(x), math.log2(x), math.log10(x)

# Powers
x ** y
pow(x, y, mod)  # (x^y) % mod

# Infinity
float('inf'), float('-inf')

# Custom GCD/LCM
def gcd(a, b):
    while b:
        a, b = b, a % b
    return a

def lcm(a, b):
    return a * b // gcd(a, b)
\end{lstlisting}

\subsection*{Combinatorics}
\begin{lstlisting}
from math import factorial, comb, perm

# nCr (combinations)
comb(n, r)

# nPr (permutations)
perm(n, r)

# Manual implementation
def ncr(n, r):
    if r > n: return 0
    r = min(r, n - r)
    num = den = 1
    for i in range(r):
        num *= (n - i)
        den *= (i + 1)
    return num // den

# Modular factorial
MOD = 10**9 + 7
def modfact(n):
    fact = [1] * (n + 1)
    for i in range(1, n + 1):
        fact[i] = fact[i-1] * i % MOD
    return fact
\end{lstlisting}

\subsection*{Number Theory}
\begin{lstlisting}
# Modular inverse (Fermat's little theorem)
def modinv(a, mod):
    return pow(a, mod - 2, mod)

# Sieve of Eratosthenes
def sieve(n):
    is_prime = [True] * (n + 1)
    is_prime[0] = is_prime[1] = False
    for i in range(2, int(n**0.5) + 1):
        if is_prime[i]:
            for j in range(i*i, n + 1, i):
                is_prime[j] = False
    return is_prime

# Prime factorization
def factorize(n):
    factors = []
    d = 2
    while d * d <= n:
        while n % d == 0:
            factors.append(d)
            n //= d
        d += 1
    if n > 1:
        factors.append(n)
    return factors

# Extended Euclidean Algorithm
def extgcd(a, b):
    if b == 0:
        return a, 1, 0
    g, x1, y1 = extgcd(b, a % b)
    return g, y1, x1 - (a // b) * y1
\end{lstlisting}

\subsection*{Graph - Adjacency List}
\begin{lstlisting}
from collections import defaultdict, deque

# Build graph
graph = defaultdict(list)
for _ in range(m):
    u, v = map(int, input().split())
    graph[u].append(v)
    graph[v].append(u)  # undirected

# Weighted graph
graph[u].append((v, weight))
\end{lstlisting}

\subsection*{BFS (Shortest Path)}
\begin{lstlisting}
def bfs(graph, start):
    visited = set([start])
    queue = deque([start])
    dist = {start: 0}
    
    while queue:
        node = queue.popleft()
        for neighbor in graph[node]:
            if neighbor not in visited:
                visited.add(neighbor)
                queue.append(neighbor)
                dist[neighbor] = dist[node] + 1
    return dist

# Grid BFS
def grid_bfs(grid, start):
    n, m = len(grid), len(grid[0])
    visited = [[False] * m for _ in range(n)]
    queue = deque([start])
    visited[start[0]][start[1]] = True
    
    dirs = [(0,1), (1,0), (0,-1), (-1,0)]
    while queue:
        x, y = queue.popleft()
        for dx, dy in dirs:
            nx, ny = x + dx, y + dy
            if 0 <= nx < n and 0 <= ny < m \
               and not visited[nx][ny] \
               and grid[nx][ny] != '#':
                visited[nx][ny] = True
                queue.append((nx, ny))
\end{lstlisting}

\subsection*{DFS}
\begin{lstlisting}
def dfs(graph, node, visited):
    visited.add(node)
    for neighbor in graph[node]:
        if neighbor not in visited:
            dfs(graph, neighbor, visited)

# Iterative DFS
def dfs_iterative(graph, start):
    visited = set()
    stack = [start]
    
    while stack:
        node = stack.pop()
        if node not in visited:
            visited.add(node)
            for neighbor in graph[node]:
                if neighbor not in visited:
                    stack.append(neighbor)

# Cycle detection (undirected)
def has_cycle(graph, n):
    visited = [False] * n
    
    def dfs(node, parent):
        visited[node] = True
        for neighbor in graph[node]:
            if not visited[neighbor]:
                if dfs(neighbor, node):
                    return True
            elif neighbor != parent:
                return True
        return False
    
    for i in range(n):
        if not visited[i]:
            if dfs(i, -1):
                return True
    return False
\end{lstlisting}

\subsection*{Dijkstra's Algorithm}
\begin{lstlisting}
import heapq

def dijkstra(graph, start, n):
    dist = [float('inf')] * n
    dist[start] = 0
    heap = [(0, start)]
    
    while heap:
        d, node = heapq.heappop(heap)
        if d > dist[node]:
            continue
        
        for neighbor, weight in graph[node]:
            new_dist = dist[node] + weight
            if new_dist < dist[neighbor]:
                dist[neighbor] = new_dist
                heapq.heappush(heap, 
                    (new_dist, neighbor))
    return dist
\end{lstlisting}

\subsection*{Topological Sort}
\begin{lstlisting}
from collections import deque

# Kahn's algorithm (BFS)
def topo_sort(graph, n):
    indegree = [0] * n
    for u in range(n):
        for v in graph[u]:
            indegree[v] += 1
    
    queue = deque([i for i in range(n) 
                   if indegree[i] == 0])
    result = []
    
    while queue:
        node = queue.popleft()
        result.append(node)
        for neighbor in graph[node]:
            indegree[neighbor] -= 1
            if indegree[neighbor] == 0:
                queue.append(neighbor)
    
    return result if len(result) == n \
           else []  # cycle exists

# DFS-based
def topo_dfs(graph, n):
    visited = [False] * n
    stack = []
    
    def dfs(node):
        visited[node] = True
        for neighbor in graph[node]:
            if not visited[neighbor]:
                dfs(neighbor)
        stack.append(node)
    
    for i in range(n):
        if not visited[i]:
            dfs(i)
    
    return stack[::-1]
\end{lstlisting}

\subsection*{Union-Find (Disjoint Set)}
\begin{lstlisting}
class UnionFind:
    def __init__(self, n):
        self.parent = list(range(n))
        self.rank = [0] * n
    
    def find(self, x):
        if self.parent[x] != x:
            self.parent[x] = self.find(
                self.parent[x])  # compression
        return self.parent[x]
    
    def union(self, x, y):
        px, py = self.find(x), self.find(y)
        if px == py:
            return False
        if self.rank[px] < self.rank[py]:
            px, py = py, px
        self.parent[py] = px
        if self.rank[px] == self.rank[py]:
            self.rank[px] += 1
        return True
    
    def connected(self, x, y):
        return self.find(x) == self.find(y)
\end{lstlisting}

\subsection*{Binary Search}
\begin{lstlisting}
# Find first True
def binary_search_first(left, right, check):
    while left < right:
        mid = (left + right) // 2
        if check(mid):
            right = mid
        else:
            left = mid + 1
    return left

# Find last True
def binary_search_last(left, right, check):
    while left < right:
        mid = (left + right + 1) // 2
        if check(mid):
            left = mid
        else:
            right = mid - 1
    return left

# Example: find sqrt
def sqrt_binary(n):
    left, right = 0, n
    while left < right:
        mid = (left + right + 1) // 2
        if mid * mid <= n:
            left = mid
        else:
            right = mid - 1
    return left
\end{lstlisting}

\subsection*{Dynamic Programming - Common Patterns}
\begin{lstlisting}
# Longest Increasing Subsequence O(n log n)
def lis(arr):
    from bisect import bisect_left
    dp = []
    for x in arr:
        idx = bisect_left(dp, x)
        if idx == len(dp):
            dp.append(x)
        else:
            dp[idx] = x
    return len(dp)

# Knapsack 0/1
def knapsack(weights, values, capacity):
    n = len(weights)
    dp = [[0] * (capacity + 1) 
          for _ in range(n + 1)]
    
    for i in range(1, n + 1):
        for w in range(capacity + 1):
            dp[i][w] = dp[i-1][w]
            if weights[i-1] <= w:
                dp[i][w] = max(dp[i][w],
                    dp[i-1][w-weights[i-1]] + 
                    values[i-1])
    return dp[n][capacity]

# Edit distance
def edit_dist(s1, s2):
    m, n = len(s1), len(s2)
    dp = [[0] * (n + 1) for _ in range(m + 1)]
    
    for i in range(m + 1):
        dp[i][0] = i
    for j in range(n + 1):
        dp[0][j] = j
    
    for i in range(1, m + 1):
        for j in range(1, n + 1):
            if s1[i-1] == s2[j-1]:
                dp[i][j] = dp[i-1][j-1]
            else:
                dp[i][j] = 1 + min(
                    dp[i-1][j],    # delete
                    dp[i][j-1],    # insert
                    dp[i-1][j-1])  # replace
    return dp[m][n]
\end{lstlisting}

\subsection*{Prefix Sum \& Difference Array}
\begin{lstlisting}
# Prefix sum
prefix = [0] * (n + 1)
for i in range(n):
    prefix[i + 1] = prefix[i] + arr[i]

# Range sum [l, r]
range_sum = prefix[r + 1] - prefix[l]

# 2D prefix sum
prefix = [[0] * (m + 1) for _ in range(n + 1)]
for i in range(1, n + 1):
    for j in range(1, m + 1):
        prefix[i][j] = matrix[i-1][j-1] + \
            prefix[i-1][j] + prefix[i][j-1] - \
            prefix[i-1][j-1]

# Rectangle sum (x1,y1) to (x2,y2)
rect_sum = prefix[x2+1][y2+1] - \
    prefix[x1][y2+1] - prefix[x2+1][y1] + \
    prefix[x1][y1]

# Difference array (range updates)
diff = [0] * (n + 1)
# Add val to range [l, r]
diff[l] += val
diff[r + 1] -= val
# Reconstruct
for i in range(1, n):
    diff[i] += diff[i - 1]
\end{lstlisting}

\subsection*{Sliding Window}
\begin{lstlisting}
# Fixed size window
def max_sum_window(arr, k):
    window_sum = sum(arr[:k])
    max_sum = window_sum
    
    for i in range(k, len(arr)):
        window_sum += arr[i] - arr[i - k]
        max_sum = max(max_sum, window_sum)
    return max_sum

# Variable size window
def min_subarray_sum(arr, target):
    left = 0
    current_sum = 0
    min_len = float('inf')
    
    for right in range(len(arr)):
        current_sum += arr[right]
        
        while current_sum >= target:
            min_len = min(min_len, 
                         right - left + 1)
            current_sum -= arr[left]
            left += 1
    
    return min_len if min_len != \
           float('inf') else 0
\end{lstlisting}

\subsection*{Segment Tree}
\begin{lstlisting}
class SegmentTree:
    def __init__(self, arr):
        self.n = len(arr)
        self.tree = [0] * (4 * self.n)
        self.build(arr, 0, 0, self.n - 1)
    
    def build(self, arr, node, start, end):
        if start == end:
            self.tree[node] = arr[start]
        else:
            mid = (start + end) // 2
            self.build(arr, 2*node+1, 
                      start, mid)
            self.build(arr, 2*node+2, 
                      mid+1, end)
            self.tree[node] = self.tree[2*node+1] \
                            + self.tree[2*node+2]
    
    def update(self, node, start, end, idx, val):
        if start == end:
            self.tree[node] = val
        else:
            mid = (start + end) // 2
            if idx <= mid:
                self.update(2*node+1, start, 
                           mid, idx, val)
            else:
                self.update(2*node+2, mid+1, 
                           end, idx, val)
            self.tree[node] = self.tree[2*node+1] \
                            + self.tree[2*node+2]
    
    def query(self, node, start, end, l, r):
        if r < start or end < l:
            return 0
        if l <= start and end <= r:
            return self.tree[node]
        mid = (start + end) // 2
        return self.query(2*node+1, start, mid, l, r) \
             + self.query(2*node+2, mid+1, end, l, r)
\end{lstlisting}

\subsection*{Fenwick Tree (BIT)}
\begin{lstlisting}
class FenwickTree:
    def __init__(self, n):
        self.n = n
        self.tree = [0] * (n + 1)
    
    def update(self, i, delta):
        while i <= self.n:
            self.tree[i] += delta
            i += i & (-i)
    
    def query(self, i):
        s = 0
        while i > 0:
            s += self.tree[i]
            i -= i & (-i)
        return s
    
    def range_query(self, l, r):
        return self.query(r) - self.query(l - 1)
\end{lstlisting}

\subsection*{Trie (Prefix Tree)}
\begin{lstlisting}
class TrieNode:
    def __init__(self):
        self.children = {}
        self.is_end = False

class Trie:
    def __init__(self):
        self.root = TrieNode()
    
    def insert(self, word):
        node = self.root
        for char in word:
            if char not in node.children:
                node.children[char] = TrieNode()
            node = node.children[char]
        node.is_end = True
    
    def search(self, word):
        node = self.root
        for char in word:
            if char not in node.children:
                return False
            node = node.children[char]
        return node.is_end
    
    def starts_with(self, prefix):
        node = self.root
        for char in prefix:
            if char not in node.children:
                return False
            node = node.children[char]
        return True
\end{lstlisting}

\subsection*{Bit Manipulation}
\begin{lstlisting}
# Check if i-th bit is set
(n >> i) & 1

# Set i-th bit
n |= (1 << i)

# Clear i-th bit
n &= ~(1 << i)

# Toggle i-th bit
n ^= (1 << i)

# Count set bits
bin(n).count('1')
n.bit_count()  # Python 3.10+

# Lowest set bit
n & -n

# Remove lowest set bit
n & (n - 1)

# Check power of 2
n > 0 and (n & (n - 1)) == 0

# Iterate over all subsets
for mask in range(1 << n):
    subset = [i for i in range(n) 
              if mask & (1 << i)]

# XOR tricks
# a ^ a = 0
# a ^ 0 = a
# Find unique: reduce(lambda x,y: x^y, arr)
\end{lstlisting}

\subsection*{Matrix Operations}
\begin{lstlisting}
# Matrix multiplication
def matmul(A, B):
    n, m, p = len(A), len(A[0]), len(B[0])
    C = [[0] * p for _ in range(n)]
    for i in range(n):
        for j in range(p):
            for k in range(m):
                C[i][j] += A[i][k] * B[k][j]
    return C

# Matrix exponentiation (mod)
def matpow(M, n, mod):
    size = len(M)
    result = [[1 if i==j else 0 
               for j in range(size)] 
              for i in range(size)]
    
    while n > 0:
        if n & 1:
            result = matmul_mod(result, M, mod)
        M = matmul_mod(M, M, mod)
        n >>= 1
    return result

def matmul_mod(A, B, mod):
    n = len(A)
    C = [[0] * n for _ in range(n)]
    for i in range(n):
        for j in range(n):
            for k in range(n):
                C[i][j] = (C[i][j] + 
                          A[i][k] * B[k][j]) % mod
    return C

# Fibonacci using matrix exp
# F(n) = [[1,1],[1,0]]^n
\end{lstlisting}

\subsection*{Miscellaneous Tips}
\begin{lstlisting}
# Fast input for large datasets
import sys
input = sys.stdin.readline

# Recursion limit
sys.setrecursionlimit(10**6)

# Deep copy
from copy import deepcopy
new_list = deepcopy(old_list)

# Iterator tools
from itertools import *
# permutations, combinations
# accumulate, product, chain

# Lambda sorting
arr.sort(key=lambda x: (-x[0], x[1]))

# All/Any
all(x > 0 for x in arr)
any(x > 0 for x in arr)

# Zip
for a, b in zip(list1, list2):
    pass

# Enumerate
for i, val in enumerate(arr):
    pass

# Custom comparison
from functools import cmp_to_key
arr.sort(key=cmp_to_key(compare_fn))
\end{lstlisting}

\subsection*{Common Pitfalls}
\begin{lstlisting}
# Integer division
a // b  # floor division
-7 // 3  # -3 (not -2)

# Modulo with negative
(-5) % 3  # 1 (not -2)

# List multiplication pitfall
matrix = [[0] * m] * n  # WRONG!
# All rows reference same list

# Correct way
matrix = [[0] * m for _ in range(n)]

# Float comparison
abs(a - b) < 1e-9

# String immutability
s = "abc"
s[0] = 'd'  # ERROR
s = 'd' + s[1:]  # OK
# Use list for mutable strings
\end{lstlisting}

\end{multicols*}
\end{document}
