% Section: Best procedure for each problem
\section{Problem procedure and contest workflow}

This section summarizes a concise, repeatable procedure to approach each contest problem efficiently.

\begin{enumerate}
  \item \textbf{Quick read (1--2 minutes)}: Read statement carefully. Identify input/output formats and required output exactly.
  \item \textbf{Classify and estimate (1--3 minutes)}: Determine problem type (graphs, math, DP, greedy, flow, geometry, data structures). Estimate difficulty and a rough solution approach and complexity.
  \item \textbf{Decide whether to attempt now}
    \begin{itemize}
      \item Solve immediately if it looks easy or you already know the pattern.
      \item Skip and mark for later if it looks hard or needs long derivation.
    \end{itemize}
  \item \textbf{Concrete examples and edge cases (2--5 minutes)}: Create small examples, corner cases, and verify understanding (empty inputs, single element, max/min values).
  \item \textbf{Design outline (5--15 minutes)}: Write a clear plan, data structures, and algorithm steps. Record complexity and memory. If proof/argument needed, sketch correctness and invariants.
  \item \textbf{Implement carefully}
    \begin{itemize}
      \item Start from a short, tested template (fast I/O, common helpers).
      \item Implement core logic in readable chunks and name variables clearly.
      \item Avoid clever one-liners that are hard to debug under time pressure.
    \end{itemize}
  \item \textbf{Test locally (2--8 minutes)}: Run sample tests, your handcrafted examples, and stress small random cases if possible. Check edge cases (overflows, boundary indices, integer division, floating precision, modulo behavior).
  \item \textbf{Optimize only if necessary}
    \begin{itemize}
      \item If complexity is borderline, identify hotspots and optimize (use faster data structures, avoid repeated work, precompute).
      \item Prefer simple, robust fixes (change algorithmic approach if needed).
    \end{itemize}
  \item \textbf{Verify and submit}
    \begin{itemize}
      \item Re-check input parsing and exact output formatting.
      \item Comment out debug prints and run final tests.
      \item Submit when confident; if WA, read feedback and debug quickly (reproduce failing case locally if possible).
    \end{itemize}
  \item \textbf{Time management and team strategy}
    \begin{itemize}
      \item Track time per problem; don’t spend excessive time on a single problem early.
      \item Coordinate with teammates (divide/parallelize work, let one implement while another tests or reads other problems).
      \item Keep a visible board of which problems are attempted, solved, or to revisit.
    \end{itemize}
\end{enumerate}

\subsection*{Quick checklist before submitting}
\begin{itemize}
  \item Correct handling of limits and types (int vs float, long long, big integers).
  \item Off-by-one and index orientation.
  \item Exact output formatting (extra spaces/newlines).
  \item Performance within time/memory limits.
  \item No remaining debug prints.
\end{itemize}

\subsection*{Common patterns and hints}
\begin{itemize}
  \item Try greedy first for constructive problems; if it fails, look for DP/graph framing.
  \item Reduce to shortest paths / flow for routing and matching problems.
  \item For counting/number theory, consider modular arithmetic and combinatorics shortcuts.
  \item Use Union-Find for connectivity under merges; use binary search on answer when monotone.
  \item When stuck, simplify constraints or solve a special case to gain insight.
\end{itemize}

% End of section
