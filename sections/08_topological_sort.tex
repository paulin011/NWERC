\section{Topological Sort}

\textbf{Description:} Linear ordering of vertices in a DAG (Directed Acyclic Graph) such that for every edge u→v, u comes before v. Used for task scheduling, course prerequisites, build systems. Time: O(V+E).

\subsection{Kahn's Algorithm (BFS-based)}
\textbf{Advantages:} Detects cycles, can process nodes level by level.

\begin{lstlisting}
from collections import deque

def topo_sort(graph, n):
    # Count incoming edges for each node
    indegree = [0] * n
    for u in range(n):
        for v in graph[u]:
            indegree[v] += 1
    
    # Start with nodes having no dependencies
    queue = deque([i for i in range(n) 
                   if indegree[i] == 0])
    result = []
    
    while queue:
        node = queue.popleft()
        result.append(node)
        
        # Remove this node from graph
        for neighbor in graph[node]:
            indegree[neighbor] -= 1
            
            # If neighbor has no more dependencies
            if indegree[neighbor] == 0:
                queue.append(neighbor)
    
    # If not all nodes processed, cycle exists
    return result if len(result) == n else []
\end{lstlisting}

\subsection{DFS-based Topological Sort}
\textbf{Advantages:} Simpler code, uses less space.

\begin{lstlisting}
def topo_dfs(graph, n):
    visited = [False] * n
    stack = []
    
    def dfs(node):
        visited[node] = True
        
        # Visit all neighbors first
        for neighbor in graph[node]:
            if not visited[neighbor]:
                dfs(neighbor)
        
        # Add to stack after visiting all descendants
        stack.append(node)
    
    # Process all components
    for i in range(n):
        if not visited[i]:
            dfs(i)
    
    # Reverse stack gives topological order
    return stack[::-1]
\end{lstlisting}
