\section{Matrix Operations}

\textbf{Description:} Matrix operations for DP optimization, graph algorithms, and recurrence relations.

\subsection{Matrix Multiplication}
\begin{lstlisting}
# Standard matrix multiplication - O(n^3)
def matmul(A, B):
    n, m, p = len(A), len(A[0]), len(B[0])
    C = [[0] * p for _ in range(n)]
    
    for i in range(n):
        for j in range(p):
            for k in range(m):
                C[i][j] += A[i][k] * B[k][j]
    
    return C

# With modulo
def matmul_mod(A, B, mod):
    n = len(A)
    C = [[0] * n for _ in range(n)]
    
    for i in range(n):
        for j in range(n):
            for k in range(n):
                C[i][j] = (C[i][j] + 
                          A[i][k] * B[k][j]) % mod
    
    return C
\end{lstlisting}

\subsection{Matrix Exponentiation}
\textbf{Description:} Compute M\textsuperscript{n} in O(k\textsuperscript{3} log n) where k is matrix dimension. Used for solving linear recurrences efficiently.

\begin{lstlisting}
def matpow(M, n, mod):
    size = len(M)
    
    # Identity matrix
    result = [[1 if i==j else 0 
               for j in range(size)] 
              for i in range(size)]
    
    # Binary exponentiation
    while n > 0:
        if n & 1:
            result = matmul_mod(result, M, mod)
        M = matmul_mod(M, M, mod)
        n >>= 1
    
    return result

# Example: Fibonacci using matrix exponentiation
# F(n) = [[1,1],[1,0]]^n
def fibonacci(n, mod):
    if n == 0: return 0
    if n == 1: return 1
    
    M = [[1, 1], [1, 0]]
    result = matpow(M, n - 1, mod)
    return result[0][0]

# Linear recurrence: a(n) = c1*a(n-1) + c2*a(n-2) + ...
# Build transition matrix and use matrix exponentiation
def linear_recurrence(coeffs, init, n, mod):
    k = len(coeffs)
    
    # Transition matrix
    # [a(n), a(n-1), ..., a(n-k+1)]
    M = [[0] * k for _ in range(k)]
    M[0] = coeffs  # First row
    for i in range(1, k):
        M[i][i-1] = 1  # Identity for shifting
    
    # Initial state vector
    state = init[::-1]  # Reverse order
    
    if n < k:
        return init[n]
    
    # M^(n-k+1)
    result_matrix = matpow(M, n - k + 1, mod)
    
    # Multiply with initial state
    result = 0
    for i in range(k):
        result = (result + result_matrix[0][i] * state[i]) % mod
    
    return result
\end{lstlisting}
